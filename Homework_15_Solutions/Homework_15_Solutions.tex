\documentclass{article}
\usepackage[utf8]{inputenc}
% are all of these packages really necessary?
% no.
% i'm just too lazy to only grab the packages i want for a specific
% document, so i just glob all of my most commonly used packages together
% this is bad practice.
\usepackage{amsmath,amsthm,amssymb,amsfonts, fancyhdr, color, comment, graphicx, environ, mdframed, soul, calc, enumitem, mdframed, xcolor, geometry, empheq, mathtools, tikz, pgfplots, caption, subcaption, hyperref}

\usetikzlibrary{external}
\tikzexternalize[prefix=tikz/,optimize command away=\includepdf]

%tikzpicture
\usepackage{tikz}
\usepackage{scalerel}
\usepackage{pict2e}
\usepackage{tkz-euclide}
\usetikzlibrary{calc}
\usetikzlibrary{patterns,arrows.meta}
\usetikzlibrary{shadows}
\usetikzlibrary{external}

%pgfplots
\usepackage{pgfplots}
\pgfplotsset{compat=newest}
\usepgfplotslibrary{statistics}
\usepgfplotslibrary{fillbetween}
\usepgfplotslibrary{polar}

\tikzset{external/export=true}
\pgfplotsset{
    standard/.style={
    axis line style = thick,
    trig format=rad,
    enlargelimits,
    axis x line=middle,
    axis y line=middle,
    enlarge x limits=0.15,
    enlarge y limits=0.15,
    every axis x label/.style={at={(current axis.right of origin)},anchor=north west},
    every axis y label/.style={at={(current axis.above origin)},anchor=south east}
    }
}
\newcommand*\widefbox[1]{\fbox{\hspace{2em}#1\hspace{2em}}}
% Command "alignedbox{}{}" for a box within an align environment
% Source: http://www.latex-community.org/forum/viewtopic.php?f=46&t=8144
\newlength\dlf  % Define a new measure, dlf
\newcommand\alignedbox[2]{
% Argument #1 = before & if there were no box (lhs)
% Argument #2 = after & if there were no box (rhs)
&  % Alignment sign of the line
{
\settowidth\dlf{$\displaystyle #1$}  
    % The width of \dlf is the width of the lhs, with a displaystyle font
\addtolength\dlf{\fboxsep+\fboxrule}  
    % Add to it the distance to the box, and the width of the line of the box
\hspace{-\dlf}  
    % Move everything dlf units to the left, so that & #1 #2 is aligned under #1 & #2
\boxed{#1 #2}
    % Put a box around lhs and rhs
}
}

\hypersetup{
    colorlinks=true,
    linkcolor=blue,
    filecolor=magenta,      
    urlcolor=cyan,
    pdftitle={Homework 15 Solutions},
    pdfpagemode=UseOutlines,
    bookmarksopen=true,
    pdfauthor={Christina Phan}
}
\newcommand{\lrp}[1]{\left( #1 \right)}
\newcommand{\abs}[1]{\left\vert #1 \right\vert}
\newcommand{\lra}[1]{\left\langle #1 \right\rangle}
\newcommand{\lrb}[1]{\left[ #1 \right]}
\newcommand{\norm}[1]{\left\lVert #1 \right\rVert}
\newcommand{\iintR}[0]{\iint\limits_{R}}
\renewcommand{\u}[0]{\mathbf{u}}
\renewcommand{\i}[0]{\mathbf{i}}
\renewcommand{\j}[0]{\mathbf{j}}
\renewcommand{\k}[0]{\mathbf{k}}
\newcommand{\T}[0]{\mathbf{T}}
\newcommand{\N}[0]{\mathbf{N}}
\newcommand{\B}[0]{\mathbf{B}}
\renewcommand{\r}[0]{\mathbf{r}}
\renewcommand{\a}[0]{\mathbf{a}}
\renewcommand{\v}[0]{\mathbf{v}}
\newcommand{\F}[0]{\mathbf{F}}
\newcommand{\eqq}[0]{\stackrel{?}{=}}

\geometry{letterpaper, portrait, margin=1in}
\renewcommand{\footrulewidth}{0.8pt}
\setlength\parindent{0pt}
\pagestyle{fancy}
\lhead{Christina Phan}
\rhead{MAT 21D} 
\chead{\textbf{Homework 15 Solutions}}

\newcommand{\Solution}{\textit{Solution}}
\pgfplotsset{compat=1.18}
\begin{document}

\phantomsection
\addcontentsline{toc}{section}{Problem 1}\textbf{Problem 1}

Consider the vector field $\displaystyle\F(x,y,z)=\lra{\frac{-y}{x^2+y^2},\frac{x}{x^2+y^2},0}$.

\phantomsection
\addcontentsline{toc}{subsection}{1(a)}\textbf{(a)} Show that $\F$ satisfies the ``component test" for conservative vector fields.

\Solution

Recall that the ``component test" states that if $\F$ is a conservative vector field then
\begin{align*}
    \frac{\partial M}{\partial y}=\frac{\partial N}{\partial x},\hspace{2em}\frac{\partial M}{\partial z}=\frac{\partial P}{\partial x},\hspace{2em} \text{and}\hspace{2em}\frac{\partial N}{\partial z}=\frac{\partial P}{\partial y}.
\end{align*}
Since $\displaystyle\F(x,y,z)=\lra{\frac{-y}{x^2+y^2},\frac{x}{x^2+y^2},0}$, \begin{align*}
    M&=\frac{-y}{x^2+y^2}\\
    N&=\frac{x}{x^2+y^2}\\
    P&=0
\end{align*}
Let's do the ``component test".

\phantomsection
\addcontentsline{toc}{subsubsection}{M and N}
For $\displaystyle \frac{\partial M}{\partial y}$ and $\displaystyle\frac{\partial N}{\partial x}$,
\begin{align*}
    \frac{\partial M}{\partial y}&=\frac{-(x^2+y^2)-2y(-y)}{(x^2+y^2)^2}=\frac{-x^2-y^2+2y^2}{(x^2+y^2)^2}=\frac{-x^2+y^2}{(x^2+y^2)^2}\\
    \frac{\partial N}{\partial x}&=
    \frac{(x^2+y^2)-2x(x)}{(x^2+y^2)^2}=\frac{x^2+y^2-2x^2}{(x^2+y^2)^2}=\frac{-x^2+y^2}{(x^2+y^2)^2}
\end{align*}
Since $\displaystyle \frac{\partial M}{\partial y}=\frac{-x^2+y^2}{(x^2+y^2)^2}$ and $\displaystyle \frac{\partial N}{\partial x}=\frac{-x^2+y^2}{(x^2+y^2)^2}$, $\displaystyle \frac{\partial M}{\partial y}=\frac{\partial N}{\partial x}$.

\phantomsection
\addcontentsline{toc}{subsubsection}{M and P}
For $\displaystyle \frac{\partial M}{\partial z}$ and $\displaystyle\frac{\partial P}{\partial x}$,
\begin{align*}
     \frac{\partial M}{\partial z}&=0\\
     \frac{\partial P}{\partial x}&=
     0
\end{align*}
Since $\displaystyle \frac{\partial M}{\partial z}=0$ and $\displaystyle \frac{\partial P}{\partial x}=0$, $\displaystyle \frac{\partial M}{\partial z}=\frac{\partial P}{\partial x}$

\phantomsection
\addcontentsline{toc}{subsubsection}{N and P}
For $\displaystyle \frac{\partial N}{\partial z}$ and $\displaystyle\frac{\partial P}{\partial y}$,
\begin{align*}
     \frac{\partial N}{\partial z}&=0\\
     \frac{\partial P}{\partial y}&=0
\end{align*}
Since $\displaystyle \frac{\partial N}{\partial z}=0$ and $\displaystyle\frac{\partial P}{\partial y}=0$, $\displaystyle \frac{\partial N}{\partial z}=\frac{\partial P}{\partial y}$.

Since $\displaystyle \frac{\partial M}{\partial y}=\frac{\partial N}{\partial x}$, $\displaystyle \frac{\partial M}{\partial z}=\frac{\partial P}{\partial x}$, and $\displaystyle \frac{\partial N}{\partial z}=\frac{\partial P}{\partial y}$, the vector field $\F$ satisfies the ``component test" for conservative vector fields.

\qed
\newpage
\phantomsection
\addcontentsline{toc}{subsection}{1(b)}\textbf{(b)} Show, however, that $\F$ is not conservative by directly evaluating $\displaystyle \oint_C \F\cdot d\r$, where $C$ is the unit circle in the $xy$-plane.

\Solution

We need to find a $\r(t)$ to evaluate our integral. Since there is a $x^2+y^2$, let's think about using a circle. Circles... let's use the $x$ and $y$ parts of our good old pal the helix, $\r(t)=\lra{a\cos t, a\sin t, t}$! Since $C$ is a unit circle, the radius is $1$. Therefore, our $\r$ must be $\r(t)=\lra{\cos t,\sin t}$ where $0\leq t \leq 2\pi$ since we're going around the full circle.

If $\r(t)=\lra{\cos t,\sin t}$,
\begin{align*}
    \F\lrp{\r(t)}&=\lra{\frac{-\sin t}{\cos^2 t +\sin^2 t},\frac{\cos t}{\cos ^2 t + \sin ^2 t},0}=\lra{-\sin t,\cos t}\tag{$\sin^2 t + \cos ^2 t=1$, we can drop $z$ since $z=0$}\\
    \r'(t)&=\lra{-\sin t, \cos t}
\end{align*}
Let's evaluate our integral.
\begin{align*}
    \oint_C \F\cdot d\r&=\int_0^{2\pi}\lra{-\sin t,\cos t}\cdot \lra{-\sin t, \cos t}\,dt\\
    &=\int_0^{2\pi} \sin^2 t + \cos ^2 t\,dt\\
    &=\int_0^{2\pi} 1\,dt\tag{$\sin^2 t + \cos ^2 t$}\\
    &=\lrb{t}_0^{2\pi}\\
    &=2\pi
\end{align*}
For conservative vector fields, $\displaystyle  \oint_C \F\cdot d\r=0$ for \textbf{any} loop $C$. Since $\displaystyle  \oint_C \F\cdot d\r=2\pi$ where $C$ is the unit circle in the $xy$-plane, the vector field $\F$ must not be conservative since for conservative vector fields since $0\neq 2\pi$.
\qed

\phantomsection
\addcontentsline{toc}{subsection}{1(c)}\textbf{(c)} Why is this not a counterexample to the test for conservative fields? (Consider the domain on which $F$ is defined.

\Solution

 The ``component test" requires that the domain of $\F$ is simply connected (no holes). This is not a counterexample to the test for conservative fields because the domain is not simply connected. That is, there is a hole at the $z$-axis. Since $\displaystyle \F(x,y,z)=\lra{\frac{-y}{x^2+y^2},\frac{x}{x^2+y^2},0}$, we know $x^2+y^2\neq 0$ since we cannot divide by $0$. Since $x^2+y^2\neq 0$, $(0,0,z)$ is not in the domain of $\F$, so the entire $z$-axis is invalid. That means, we have a hole in our domain, so the ``component test" is not appropriate to use here.

\phantomsection
\addcontentsline{toc}{section}{Problem 2}\textbf{Problem 2}

Determine whether the vector field is conservative:

\phantomsection
\addcontentsline{toc}{subsection}{2(a)}\textbf{(a)} $\F(x,y,z)=\lra{y\sin z, x\sin z, xy\cos z}$

\Solution

Recall that the ``component test" states that if $\F$ is a conservative vector field then
\begin{align*}
    \frac{\partial M}{\partial y}=\frac{\partial N}{\partial x},\hspace{2em}\frac{\partial M}{\partial z}=\frac{\partial P}{\partial x},\hspace{2em} \text{and}\hspace{2em}\frac{\partial N}{\partial z}=\frac{\partial P}{\partial y}.
\end{align*}
Since $\displaystyle\F(x,y,z)=\lra{y\sin z, x\sin z, xy\cos z}$, \begin{align*}
    M&=y\sin z\\
    N&=x\sin z\\
    P&=xy\cos z
\end{align*}
Let's use the ``component test".

\phantomsection
\addcontentsline{toc}{subsubsection}{M and N}
For $\displaystyle \frac{\partial M}{\partial y}$ and $\displaystyle\frac{\partial N}{\partial x}$,
\begin{align*}
    \frac{\partial M}{\partial y}&= \sin z\\
    \frac{\partial N}{\partial x}&=
    \sin z
\end{align*}
Since $\displaystyle \frac{\partial M}{\partial y}=\sin z$ and $\displaystyle\frac{\partial N}{\partial x}=\sin z$, $\displaystyle \frac{\partial M}{\partial y}=\displaystyle\frac{\partial N}{\partial x}$.

\phantomsection
\addcontentsline{toc}{subsubsection}{M and P}
For $\displaystyle\frac{\partial M}{\partial z}$ and $\displaystyle\frac{\partial P}{\partial x}$
\begin{align*}
    \frac{\partial M}{\partial z}&=y\cos z\\
    \frac{\partial P}{\partial x}&=
    y\cos z
\end{align*}
Since $\displaystyle\frac{\partial M}{\partial z}=y\cos z$ and $\displaystyle\frac{\partial P}{\partial x}=y\cos z$,  $\displaystyle\frac{\partial M}{\partial z}=\displaystyle\frac{\partial P}{\partial x}$.

\phantomsection
\addcontentsline{toc}{subsubsection}{N and P}
For $\displaystyle \frac{\partial N}{\partial z}$ and $\displaystyle\frac{\partial P}{\partial y}$,
\begin{align*}
     \frac{\partial N}{\partial z}&=x\cos z\\
     \frac{\partial P}{\partial y}&=x\cos z
\end{align*}
Since $\displaystyle \frac{\partial N}{\partial z}=x\cos z$ and $\displaystyle\frac{\partial P}{\partial y}=x\cos z$, $\displaystyle \frac{\partial N}{\partial z}=\displaystyle\frac{\partial P}{\partial y}$.

Since $\displaystyle \frac{\partial M}{\partial y}=\frac{\partial N}{\partial x}$, $\displaystyle \frac{\partial M}{\partial z}=\frac{\partial P}{\partial x}$, and $\displaystyle \frac{\partial N}{\partial z}=\frac{\partial P}{\partial y}$, $\F$ satisfies the ``component test" for conservative vector fields.

\qed

\phantomsection
\addcontentsline{toc}{subsection}{2(b)}\textbf{(b)} $\F(x,y,z)=\lra{y,x+z,-y}$

\Solution

Recall that the ``component test" states that if $\F$ is a conservative vector field then
\begin{align*}
    \frac{\partial M}{\partial y}=\frac{\partial N}{\partial x},\hspace{2em}\frac{\partial M}{\partial z}=\frac{\partial P}{\partial x},\hspace{2em} \text{and}\hspace{2em}\frac{\partial N}{\partial z}=\frac{\partial P}{\partial y}.
\end{align*}
Since $\displaystyle\F(x,y,z)=\lra{y,x+z,-y}$, \begin{align*}
    M&=y\\
    N&=x+z\\
    P&=-y
\end{align*}
Our domain is simply connected because there are no holes in the domain ($y$, $x+z$, and $-y$ are all continuous).

Let's use the ``component test".

Check out this sick trick I do :D

\phantomsection
\addcontentsline{toc}{subsubsection}{N and P}
For $\displaystyle \frac{\partial N}{\partial z}$ and $\displaystyle\frac{\partial P}{\partial y}$,
\begin{align*}
     \frac{\partial N}{\partial z}&=1\\
     \frac{\partial P}{\partial y}&= -1
\end{align*}
\hl{\textbf{stop here :)}}

Since $\displaystyle \frac{\partial N}{\partial z}\neq\frac{\partial P}{\partial y}$, $\F$ does not satisfy the ``component test" for conservative vector fields.

\qed

\phantomsection
\addcontentsline{toc}{subsection}{2(c)}\textbf{(c)} $\F(x,y,z)=\lra{z+y,z,y+z}$

\Solution

Recall that the ``component test" states that if $\F$ is a conservative vector field then
\begin{align*}
    \frac{\partial M}{\partial y}=\frac{\partial N}{\partial x},\hspace{2em}\frac{\partial M}{\partial z}=\frac{\partial P}{\partial x},\hspace{2em} \text{and}\hspace{2em}\frac{\partial N}{\partial z}=\frac{\partial P}{\partial y}.
\end{align*}
Since $\F(x,y,z)=\lra{z+y,z,y+z}$, \begin{align*}
    M&=z+y\\
    N&=z\\
    P&=y+z
\end{align*}
Let's use the ``component test".

Check out this sick trick I do :D

\phantomsection
\addcontentsline{toc}{subsubsection}{M and N}
For $\displaystyle \frac{\partial M}{\partial y}\neq\frac{\partial N}{\partial x}$,
\begin{align*}
    \frac{\partial M}{\partial y}&=1\\
    \frac{\partial N}{\partial x}&=0
\end{align*}
\hl{\textbf{stop here :)}}

Since $\displaystyle \frac{\partial M}{\partial y}\neq\frac{\partial N}{\partial x}$, $\F$ does not satisfy the ``component test" for conservative vector fields.
\qed

\phantomsection
\addcontentsline{toc}{section}{Problem 3}\textbf{Problem 3}

Show that the differential form in the integral is exact and evaluate the integral:

\phantomsection
\addcontentsline{toc}{subsection}{3(a)}\textbf{(a)} $\displaystyle\int_{(0,0,0)}^{(2,3,-6)}2x\,dx+2y\,dy+2z\,dz$

\Solution

Recall that a differential form is exact if and only if
\begin{align*}
    \frac{\partial M}{\partial y}=\frac{\partial N}{\partial x},\hspace{2em}\frac{\partial M}{\partial z}=\frac{\partial P}{\partial x},\hspace{2em} \text{and}\hspace{2em}\frac{\partial N}{\partial z}=\frac{\partial P}{\partial y}.
\end{align*}
Since $\displaystyle\F(x,y,z)=\lra{2x,2y,2z}$, \begin{align*}
    M&=2x\\
    N&=2y\\
    P&=2z
\end{align*}
Let's check if the differential form in the integral is exact.

\phantomsection
\addcontentsline{toc}{subsubsection}{M and N}
For $\displaystyle \frac{\partial M}{\partial y}$ and $\displaystyle \frac{\partial N}{\partial x}$,
\begin{align*}
    \frac{\partial M}{\partial y}&=0\\
    \frac{\partial N}{\partial x}&=0
\end{align*}
Since $\displaystyle \frac{\partial M}{\partial y}=0$ and $\displaystyle \frac{\partial N}{\partial x}=0$, $\displaystyle \frac{\partial M}{\partial y}=\displaystyle \frac{\partial N}{\partial x}$.

\phantomsection
\addcontentsline{toc}{subsubsection}{M and P}
For $\displaystyle \frac{\partial M}{\partial z}$ and $\displaystyle\frac{\partial P}{\partial x}$,
\begin{align*}
    \frac{\partial M}{\partial z}&=0\\
    \frac{\partial P}{\partial x}&=0
\end{align*}
Since $\displaystyle \frac{\partial M}{\partial z}=0$ and $\displaystyle\frac{\partial P}{\partial x}=0$, $\displaystyle \frac{\partial M}{\partial z}=\displaystyle\frac{\partial P}{\partial x}$.

\phantomsection
\addcontentsline{toc}{subsubsection}{N and P}
For $\displaystyle \frac{\partial N}{\partial z}$ and $\displaystyle\frac{\partial P}{\partial y}$,
\begin{align*}
    \frac{\partial N}{\partial z}&=0\\
    \frac{\partial P}{\partial y}&=0
\end{align*}
Since $\displaystyle \frac{\partial N}{\partial z}=0$ and $\displaystyle\frac{\partial P}{\partial y}=0$, $\displaystyle \frac{\partial N}{\partial z}=\displaystyle\frac{\partial P}{\partial y}$.

Since $\displaystyle \frac{\partial M}{\partial y}=\frac{\partial N}{\partial x}$, $\displaystyle \frac{\partial M}{\partial z}=\frac{\partial P}{\partial x}$, and $\displaystyle \frac{\partial N}{\partial z}=\frac{\partial P}{\partial y}$, $\F$ is exact.

Let's start finding $f(x,y,z)$ so we can evaluate the integral.

Since $\displaystyle \frac{\partial f}{\partial x}=2x$,
\begin{align*}
    f(x,y,z)&=x^2+g(y,z)
\end{align*}
Since $\displaystyle \frac{\partial f}{\partial y}=2y$,
\begin{align*}
    \frac{\partial f}{\partial y}= 0 + \frac{\partial g}{\partial y}&= 2y\\
    \implies \frac{\partial g}{\partial y}&= 2y\\
    \implies f(x,y,z)&=x^2+y^2  +h(z)
\end{align*}
Finally, since $\displaystyle \frac{\partial f}{\partial z}=2z$,
\begin{align*}
\frac{\partial f}{\partial z}=0+0+\frac{\partial h}{\partial z}&=2z\\
\implies \frac{\partial h}{\partial z}&=2z\\
\implies f(x,y,z)&=x^2+y^2+z^2 \tag{let $C=0$ aka ignore $C$}
\end{align*}
Since $f(x,y,z)=x^2+y^2+z^2$,
\begin{align*}
    \int_{(0,0,0)}^{(2,3,-6)}2x\,dx+2y\,dy+2z\,dz&=\lrb{f(x,y,z)}_{(0,0,0)}^{(2,3,-6)}\\
    &=f(2,3,-6)-f(0,0,0)\\
    &=\lrp{2^2+3^2+(-6)^2}-\lrp{0^2+0^2+0^2}\\
    &=\boxed{49}
\end{align*}
\newpage
\phantomsection
\addcontentsline{toc}{subsection}{3(b)}\textbf{(b)} $\displaystyle \int_{(0,0,0)}^{(3,3,1)}2xy\,dx+(x^2-z^2)\,dy-2yz\,dz$

\Solution

Recall that a differential form is exact if and only if
\begin{align*}
    \frac{\partial M}{\partial y}=\frac{\partial N}{\partial x},\hspace{2em}\frac{\partial M}{\partial z}=\frac{\partial P}{\partial x},\hspace{2em} \text{and}\hspace{2em}\frac{\partial N}{\partial z}=\frac{\partial P}{\partial y}.
\end{align*}
Since $\displaystyle\F(x,y,z)=\lra{2xy,x^2-z^2,-2yz}$, \begin{align*}
    M&=2xy\\
    N&=x^2-z^2\\
    P&=-2yz
\end{align*}
Let's check if the differential form in the integral is exact.

\phantomsection
\addcontentsline{toc}{subsubsection}{M and N}
For $\displaystyle \frac{\partial M}{\partial y}$ and $\displaystyle\frac{\partial N}{\partial x}$,
\begin{align*}
    \frac{\partial M}{\partial y}&= 2x\\
    \frac{\partial N}{\partial x}&=2x
\end{align*}
Since $\displaystyle \frac{\partial M}{\partial y}=2x$ and $\displaystyle\frac{\partial N}{\partial x}=2x$, $\displaystyle \frac{\partial M}{\partial y}=\frac{\partial N}{\partial x}$.

\phantomsection
\addcontentsline{toc}{subsubsection}{M and P}
For $\displaystyle \frac{\partial M}{\partial z}$ and $\displaystyle\frac{\partial P}{\partial x}$,
\begin{align*}
    \frac{\partial M}{\partial z}&=0\\ \frac{\partial P}{\partial x}&=0
\end{align*}
Since $\displaystyle \frac{\partial M}{\partial z}=0$ and $\displaystyle\frac{\partial P}{\partial x}=0$, $\displaystyle \frac{\partial M}{\partial z}=\displaystyle\frac{\partial P}{\partial x}$.

\phantomsection
\addcontentsline{toc}{subsubsection}{N and P}
For $\displaystyle \frac{\partial N}{\partial z}$ and $\displaystyle \frac{\partial P}{\partial y}$,
\begin{align*}
    \frac{\partial N}{\partial z}&=-2z\\
    \frac{\partial P}{\partial y}&=-2z
\end{align*}
Since $\displaystyle \frac{\partial N}{\partial z}=-2z$ and $\displaystyle \frac{\partial P}{\partial y}=-2z$,  $\displaystyle \frac{\partial N}{\partial z}=\displaystyle \frac{\partial P}{\partial y}$.

Since $\displaystyle \frac{\partial M}{\partial y}=\frac{\partial N}{\partial x}$, $\displaystyle \frac{\partial M}{\partial z}=\frac{\partial P}{\partial x}$, and $\displaystyle \frac{\partial N}{\partial z}=\frac{\partial P}{\partial y}$, $\F$ is exact.

Let's start finding $f(x,y,z)$ so we can evaluate the integral.

Since $\displaystyle \frac{\partial f}{\partial x}=2xy$
\begin{align*}
   f(x,y,z)&=x^2y+g(y,z)
\end{align*}
Since $\displaystyle \frac{\partial f}{\partial y}=x^2-z^2$,
\begin{align*}
    \frac{\partial f}{\partial y}=x^2+\frac{\partial g}{\partial y}&=x^2-z^2\\
    \implies \frac{\partial g}{\partial y}&=-z^2\\
    \implies f(x,y,z)&= x^2y -yz^2+h(z)
\end{align*}
Since $\displaystyle \frac{\partial f}{\partial z}=-2yz$,
\begin{align*}
    \frac{\partial f}{\partial z}=0-2yz+\frac{\partial h}{\partial z}&=-2yz\\
    \implies \frac{\partial h}{\partial z}&=0\\
    \implies f(x,y,z)&=x^2y-yz^2\tag{let $C=0$ aka ignore $C$}
\end{align*}
Since $f(x,y,z)=x^2y-yz^2$,
\begin{align*}
    \int_{(0,0,0)}^{(3,3,1)}2xy\,dx+(x^2-z^2)\,dy-2yz\,dz&=\lrb{f(x,y,z)}_{(0,0,0)}^{(3,3,1)}\\
    &=f(3,3,1)-f(0,0,0)\\
    &=\Big((3)^2(3)-(3)(1)^2\Big)-\Big((0)^2(0)-(0)(0)^2\Big)\\
    &=\lrp{27 - 3}-\lrp{0-0}\\
    &=\boxed{24}
\end{align*}
\phantomsection
\addcontentsline{toc}{section}{Problem 4}\textbf{Problem 4} 

Evaluate the integral $\displaystyle \int_C 2x\cos y\,dx -x^2\sin y\,dy$ along the path $C$:

\phantomsection

\Solution

If we can prove that if $\F$ is conservative, then there exists an $f$ such that $\F=\nabla f$.

If $\F$ is conservative, then by the ``component test",
\begin{align*}
    \frac{\partial M}{\partial y}=\frac{\partial N}{\partial x}
\end{align*}
Since $\displaystyle\F(x,y,z)=\lra{2x\cos y. -x^2\sin y}$, \begin{align*}
    M&=2x\cos y\\
    N&=-x^2\sin y\\
\end{align*}
Let's do the ``component test".
\phantomsection
\addcontentsline{toc}{subsection}{M and N}
For $\displaystyle \frac{\partial M}{\partial y}$ and $\displaystyle\frac{\partial N}{\partial x}$,
\begin{align*}
    \frac{\partial M}{\partial y}&= -2x\sin y\\
    \frac{\partial N}{\partial x}&=-2x\sin y
\end{align*}
Since $\displaystyle \frac{\partial M}{\partial y}=-2x\sin y$ and $\frac{\partial N}{\partial x}=-2x\sin y$, $\displaystyle \frac{\partial M}{\partial y}=\frac{\partial N}{\partial x}$, so $\F$ is conservative.

Let's find $f(x,y)$.

Since $\displaystyle \frac{\partial f}{\partial x}=2x\cos y$,
\begin{align*}
    f(x,y)&=x^2\cos y + g(y)
\end{align*}
Since $\displaystyle \frac{\partial f}{\partial y}=-x^2\sin y$,
\begin{align*}
    \frac{\partial f}{\partial y}=-x^2\sin y + \frac{\partial g}{\partial y}&=-x^2\sin y\\
    \implies \frac{\partial g}{\partial y}&=0\\
    \implies f(x,y)&=x^2\cos y\tag{let $C=0$ aka ignore $C$}
\end{align*}
Therefore, $\F = \nabla f$ where $f(x,y)=x^2\cos y$.

\addcontentsline{toc}{subsection}{4(a)}\textbf{(a)} $C$ is a parabola $y=(x-1)^2$ from $(1,0)$ to $(0,1)$.

\Solution

Since $\F = \nabla f$ where $f(x,y)=x^2\cos y$,
\begin{align*}
    \int_C 2x\cos y\,dx -x^2\sin y\,dy&=\lrb{f(x,y)}_{(1,0)}^{(0,1)}\\
    &=f(1,0)-f(0,1)\\
    &=\lrp{0^2\cos 1}-\lrp{1^2\cos 0}\\
    &=\lrp{0}-\lrp{1}\\
    &=\boxed{-1}
\end{align*}
\phantomsection
\addcontentsline{toc}{subsection}{4(b}\textbf{(b)} $C$ is the line segment from $(-1,\pi)$ to $(1,0)$

\Solution

Since $\F = \nabla f$ where $f(x,y)=x^2\cos y$,
\begin{align*}
    \int_C 2x\cos y\,dx -x^2\sin y\,dy&=\lrb{f(x,y)}_{(-1, \pi)}^{(1,0)}\\
    &=f(1, 0)-f(-1,\pi)\\
    &=\lrp{1^2 \cos 0}-\lrp{(-1)^2\cos \pi}\\
    &=\lrp{1}-\lrp{-1}\\
    &=\boxed{2}
\end{align*}

\phantomsection
\addcontentsline{toc}{subsection}{4(c)}\textbf{(c)} $C$ is the asteroid $\r(t)=\lra{\cos^3 t,\sin ^3 t}$, $0\leq t \leq 2\pi$

\Solution

Since $\F = \nabla f$ where $f(x,y)=x^2\cos y$,
\begin{align*}
    \int_C 2x\cos y\,dx -x^2\sin y\,dy&=\lrb{f(x,y)}_{(1,0)}^{(1,0)}\\
    &=f(1,0)-f(1,0)\\
    &=\boxed{0}\tag{or find what $f(1,0)-f(1,0)$ is yourself}
\end{align*}
\phantomsection
\addcontentsline{toc}{section}{Problem 5}\textbf{Problem 5}

How are the constants $a$, $b$, and $c$ related if the differential form\begin{equation*}
    (ay^2+2czx)\,dx+y(bx+cz)\,dy+(ay^2+cx^2)\,dz
\end{equation*}
is exact?

\Solution

If the differential form \begin{equation*}
    (ay^2+2czx)\,dx+y(bx+cz)\,dy+(ay^2+cx^2)\,dz
\end{equation*}
is exact, then  $\displaystyle \frac{\partial M}{\partial y}=\frac{\partial N}{\partial x}$, $\displaystyle \frac{\partial M}{\partial z}=\frac{\partial P}{\partial x}$, and $\displaystyle \frac{\partial N}{\partial z}=\frac{\partial P}{\partial y}$. Therefore,

\phantomsection
\addcontentsline{toc}{subsection}{M and N}
For $\displaystyle \frac{\partial M}{\partial y}=\frac{\partial N}{\partial x}$,
\begin{align*}
    \frac{\partial M}{\partial y}&=\frac{\partial N}{\partial x}\\
    2ay&=by
\end{align*}
\phantomsection
\addcontentsline{toc}{subsection}{M and P}
For $\displaystyle \frac{\partial M}{\partial z}=\frac{\partial P}{\partial x}$,
\begin{align*}
    \frac{\partial M}{\partial z}&=\frac{\partial P}{\partial x}\\
    2cx&=2cx
\end{align*}
\phantomsection
\addcontentsline{toc}{subsection}{N and P}
For $\displaystyle \frac{\partial N}{\partial z}=\frac{\partial P}{\partial y}$,
\begin{align*}
    \frac{\partial N}{\partial z}&=\frac{\partial M}{\partial y}\\
    cy&=2ay
\end{align*}
We now have the following system of equation to solve.
\begin{align*}
    2ay&=by\implies 2a=b\\
    2cx&=2cx\implies \text{nothing lol}\\
    cy&=2ay\implies c=2a
\end{align*}
In total, $a$ is our independent variable. The variables $b$ and $c$ depend on $a$ since $b=2a$ and $c=2a$. Since $b=2a$ and $c=2a$, $b=c$.

\qed
\end{document}
