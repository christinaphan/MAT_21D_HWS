\documentclass{article}
\usepackage[utf8]{inputenc}
% are all of these packages really necessary?
% no.
% i'm just too lazy to only grab the packages i want for a specific
% document, so i just glob all of my most commonly used packages together
% this is bad practice.
\usepackage{amsmath,amsthm,amssymb,amsfonts, fancyhdr, color, comment, graphicx, environ, mdframed, soul, calc, enumitem, mdframed, xcolor, geometry, empheq, mathtools, tikz, pgfplots, caption, subcaption, hyperref}

\usetikzlibrary{external}
\tikzexternalize[prefix=tikz/,optimize command away=\includepdf]

%tikzpicture
\usepackage{tikz}
\usepackage{scalerel}
\usepackage{pict2e}
\usepackage{tkz-euclide}
\usetikzlibrary{calc}
\usetikzlibrary{patterns,arrows.meta}
\usetikzlibrary{shadows}
\usetikzlibrary{external}

%pgfplots
\usepackage{pgfplots}
\pgfplotsset{compat=newest}
\usepgfplotslibrary{statistics}
\usepgfplotslibrary{fillbetween}
\usepgfplotslibrary{polar}

\tikzset{external/export=true}
\pgfplotsset{
    standard/.style={
    axis line style = thick,
    trig format=rad,
    enlargelimits,
    axis x line=middle,
    axis y line=middle,
    enlarge x limits=0.15,
    enlarge y limits=0.15,
    every axis x label/.style={at={(current axis.right of origin)},anchor=north west},
    every axis y label/.style={at={(current axis.above origin)},anchor=south east}
    }
}
\newcommand*\widefbox[1]{\fbox{\hspace{2em}#1\hspace{2em}}}
% Command "alignedbox{}{}" for a box within an align environment
% Source: http://www.latex-community.org/forum/viewtopic.php?f=46&t=8144
\newlength\dlf  % Define a new measure, dlf
\newcommand\alignedbox[2]{
% Argument #1 = before & if there were no box (lhs)
% Argument #2 = after & if there were no box (rhs)
&  % Alignment sign of the line
{
\settowidth\dlf{$\displaystyle #1$}  
    % The width of \dlf is the width of the lhs, with a displaystyle font
\addtolength\dlf{\fboxsep+\fboxrule}  
    % Add to it the distance to the box, and the width of the line of the box
\hspace{-\dlf}  
    % Move everything dlf units to the left, so that & #1 #2 is aligned under #1 & #2
\boxed{#1 #2}
    % Put a box around lhs and rhs
}
}

\hypersetup{
    colorlinks=true,
    linkcolor=blue,
    filecolor=magenta,      
    urlcolor=cyan,
    pdftitle={Homework 13 Solutions},
    pdfpagemode=UseOutlines,
    bookmarksopen=true,
    pdfauthor={Christina Phan}
}
\newcommand{\lrp}[1]{\left( #1 \right)}
\newcommand{\abs}[1]{\left\vert #1 \right\vert}
\newcommand{\lra}[1]{\left\langle #1 \right\rangle}
\newcommand{\lrb}[1]{\left[ #1 \right]}
\newcommand{\norm}[1]{\left\lVert #1 \right\rVert}
\newcommand{\iintR}[0]{\iint\limits_{R}}
\renewcommand{\u}[0]{\mathbf{u}}
\renewcommand{\i}[0]{\mathbf{i}}
\renewcommand{\j}[0]{\mathbf{j}}
\renewcommand{\k}[0]{\mathbf{k}}
\newcommand{\T}[0]{\mathbf{T}}
\newcommand{\N}[0]{\mathbf{N}}
\newcommand{\B}[0]{\mathbf{B}}
\renewcommand{\r}[0]{\mathbf{r}}
\renewcommand{\a}[0]{\mathbf{a}}
\renewcommand{\v}[0]{\mathbf{v}}
\newcommand{\F}[0]{\mathbf{F}}

\geometry{letterpaper, portrait, margin=1in}
\renewcommand{\footrulewidth}{0.8pt}
\setlength\parindent{0pt}
\pagestyle{fancy}
\lhead{Christina Phan}
\rhead{MAT 21D} 
\chead{\textbf{Homework 13 Solutions}}

\newcommand{\Solution}{\textit{Solution}}
\pgfplotsset{compat=1.18}
\begin{document}
\phantomsection
\addcontentsline{toc}{section}{Problem 1 (Parts)}\textbf{Problem 1 (Parts)}

Find the work done by the force $\F$ over the curve:

\phantomsection
\addcontentsline{toc}{subsection}{1(a)}\textbf{(a)} $\F\lrp{x,y,z}=\lra{xy,y,-yz}$, $\r(t)=\lra{t,t^2,t}$, $0\leq t\leq 1$

\Solution

Recall that work done by the force $\F$ over a curve is
\begin{align*}
    W&=\int_C M\,dx + N\,dy + P\,dz
\end{align*}
Note that this is \textit{one} of the many ways to define work. I am using the definition of work that is convenient for this problem. Plus, remembering the formula this way helps you remember the other identical formulas hahaha

If $\r(t)=\lra{t,t^2,t}$, then
\begin{align*}
    \F\lrp{\r(t)}&=\lra{(t)(t^2),t^2,-(t^2)(t)}=\lra{t^3,t^2,-t^3}\\
    \r'(t)&=\lra{1,2t,1}
\end{align*}
Let's find the work for this problem.
\begin{align*}
    W&=\int_0^1 (t^3)(1)+(t^2)(2t^3)+(-t^3)(1)\,dt\\
    &=\int_0^1 t^3 +2t^3 - t^3\,dt\\
    &=\int_0^1 2t^3\,dt\\
    &=\lrb{\frac{1}{2}t^4}_0^1\\
    &=\boxed{\frac{1}{2}}
\end{align*}

\phantomsection
\addcontentsline{toc}{subsection}{1(b)}\textbf{(b)} $\F\lrp{x,y,z}=\lra{6z,y^2,12x}$, $\displaystyle\r(t)=\lra{\sin t, \cos t, \frac{t}{6}}$, $0\leq t \leq 2\pi$

\Solution

Recall that work done by the force $\F$ over a curve is
\begin{align*}
    W&=\int_C M\,dx + N\,dy + P\,dz
\end{align*}
Note that this is \textit{one} of the many ways to define work. I am using the definition of work that is convenient for this problem. Plus, remembering the formula this way helps you remember the other identical formulas hahaha

If $\displaystyle\r(t)=\lra{\sin t, \cos t, \frac{t}{6}}$, then
\begin{align*}
    \F\lrp{\r(t)}&=\lra{6\lrp{\frac{t}{6}},\cos^2t,12\sin t}=\lra{t,\cos^2 t, 12\sin t}\\
    \r'(t)&=\lra{\cos t, -\sin t, \frac{1}{6}}
\end{align*}
Let's find the work for this problem.
\begin{align*}
    W&=\int_0^{2\pi} (t)(\cos t)+(\cos^2t)(- \sin t) + \lrp{12\sin t}\lrp{\frac{1}{6}}\,dt\\
    &=\int_0^{2\pi} t\cos t - \sin t\cos^2t + 2\sin t\,dt\\
    &=\int_0^{2\pi}t\cos t\,dt + \int_0^{2\pi}\sin t \cos ^2 t \,dt + \int_0^{2\pi} 2\sin t\,dt\\
    &u=t\hspace{2em}dv=\cos t\,dt\\
    &du=dt\hspace{2em}v=\sin t\\
    &=\lrp{\lrb{t\sin t}_0^{2\pi}-\int_0^{2\pi}\sin t\,dt}+\lrb{\frac{1}{3}\cos^3t}_0^{2\pi} +\lrb{-2\cos t}_0^{2\pi}\\
    &=\lrp{2\pi \sin 2\pi - 0\sin 0}-\lrb{-\cos t}_0^{2\pi}+\lrp{\frac{1}{3}\cos ^3 2\pi -\frac{1}{3}\cos^3 0 }+\lrp{-2 \cos 2\pi - (-2\cos 0)}\\
    &=(0-0)-\lrp{-\cos 2\pi - (-\cos 0)}+\lrp{\frac{1}{3}-\frac{1}{3}}+\lrp{-2 -(-2)}\\
    &=(0)-\lrp{-1-(-1)}+(0)+(0)\\
    &=\boxed{0}
\end{align*}

\phantomsection
\addcontentsline{toc}{subsection}{1(c)}\textbf{(c)} $\F\lrp{x,y}$ is the gradient of $f(x,y)=(x+y)^2$, counterclockwise around the circle $x^2+y^2=4$ starting and ending at $(2,0)$

\Solution

If $\F\lrp{x,y}$ is the gradient of $f(x,y)=(x+y)^2$, then
\begin{align*}
    \F &= \nabla f\\
    &=\lra{\frac{\partial f}{\partial x},\frac{\partial f}{\partial y}}\\
    &=\lra{2(x+y),2(x+y)}
\end{align*}
If we're going counterclockwise around the circle $x^2+y^2=4$ starting and ending at $(2,0)$, let's find $\r(t)$ using our good old pal the helix, $\r(t)=\lra{a \cos t,a\sin t, t}$. Since the radius of the unit circle is $2$, $\r(t)=\lra{2\cos t, 2\sin t}$.

Since we're going counterclockwise from $(2,0)$ to $(2,0)$, $0\leq t\leq 2\pi$ since at $\r(0)=\lra{2,0}$ and at $t=2\pi$, $\r\lrp{2\pi}=\lra{2,0}$.

If $\r(t)=\lra{2\cos t, 2\sin t}$,
\begin{align*}
    \F\lrp{\r(t)}&=\lra{2(2\cos t + 2\sin t),2(2\cos t + 2\sin t)}=\lra{4(\cos t + \sin t), 4(\cos t + \sin t)}\\
   \r'(t)&=\lra{-2\sin t, 2\cos t}
\end{align*}
Recall that work done by the force $\F$ over a curve is
\begin{align*}
    W&=\int_C M\,dx + N\,dy + P\,dz
\end{align*}
Note that this is \textit{one} of the many ways to define work. I am using the definition of work that is convenient for this problem. Plus, remembering the formula this way helps you remember the other identical formulas hahaha

Let's find the work for this problem.
\begin{align*}
    W&=\int_0^{2\pi} \big(4(\cos t + \sin t)\big)\lrp{-2\sin t}+\big(4(\cos t + \sin t)\big)\lrp{2\cos t}\,dt\\
    &=\int_0^{2\pi}-8\sin t\cos t - 8\sin^2 t + 8 \cos^2 t + 8\sin t \cos t\,dt\\
    &=\int_0^{2\pi} -8\sin^2 t + 8\cos^2 t\,dt\\
    &=\int_0^{2\pi} 8(\cos^2 t - \sin^2 t)\,dt\\
    &=\int_0^{2\pi}8\cos 2t\,dt\tag{$\cos^2 t - \sin^2 t=\cos 2t$}\\
    &=\lrb{4\sin 2t}_0^{2\pi}\\
    &=4\sin 4\pi - 4\sin 0\\
    &=0 - 0\\
    &=\boxed{0}
\end{align*}

\phantomsection
\addcontentsline{toc}{section}{Problem 2 (Parts)}\textbf{Problem 2 (Parts)} 

Find the flow of the velocity field $\F$ along the curve:

\phantomsection
\addcontentsline{toc}{subsection}{2(a)}\textbf{(a)} $\displaystyle\F(x,y)=\lra{\frac{x}{y+1},\frac{y}{x+1}}$, $\r(t)=\lra{t^2,t}$, $0\leq t \leq 1$

\Solution

Recall that the flow of the velocity field $\F$ along a curve is
\begin{align*}
    \text{flow}&=\int_C M\,dx + N\,dy
\end{align*}
Note that this is \textit{one} of the many ways to define work. I am using the definition of work that is convenient for this problem. Plus, remembering the formula this way helps you remember the other identical formulas hahaha

If $\r(t)=\lra{t^2,t}$, then
\begin{align*}
    \F\lrp{\r(t)}&=\lra{\frac{t^2}{t+1},\frac{t}{t^2+1}}\\
    \r'(t)&=\lra{2t,1}
\end{align*}
Let's find the flow for this problem.
\begin{align*}
    \text{flow}&=\int_0^1 \lrp{\frac{t^2}{t+1}}\lrp{2t}+\lrp{\frac{t}{t^2+1}}\lrp{1}\,dt\\
    &=\int_0^1 \frac{2t^3}{t+1}+\frac{t}{t^2+1}\,dt\\
    &=\underbrace{\int_0^1 \frac{2t^3}{t+1}\,dt}_{u} + \underbrace{\int_0^1 \frac{t}{t^2+1}\,dt}_{v}\\
    &u=t+1\implies t=u-1\hspace{2em}du=dt\\
    &u(0)=1\hspace{9em}u(1)=2\\
    &v=t^2+1\hspace{9em}dv=2t\,dt\\
    &v(0)=1\hspace{9em}v(1)=2\\
    &=\int_1^2 \frac{2(u-1)^3}{u}\,du+\frac{1}{2}\int_1^2 \frac{1}{v}\,dv\\
    &=\int_1^2 \frac{2(u^3-3u^2+3u-1)}{u}\,du+\frac{1}{2}\lrb{\ln \left |v\right|}_1^2\\
    &=2\int_1^2 (u^2 - 3u+3-\frac{1}{u}\,dt+\frac{1}{2}\lrp{\ln 2 - \ln 1}\\
    &=2\lrb{\frac{1}{3}u^3-\frac{3}{2}u^2+3u-\ln \left|u\right|}_1^2 + \frac{1}{2}\lrp{\ln 2 - 0}\\
    &=2\lrp{\lrp{\frac{1}{3}(2)^3 -\frac{3}{2}(2)^2+3(2)-\ln 2}-\lrp{\frac{1}{3}-\frac{3}{2}+3-\ln 1}}+\frac{1}{2}\ln 2\\
    &=2\lrp{\lrp{\frac{8}{3}-\ln 2}-\lrp{\frac{11}{6}}}+\frac{1}{2}\ln 2\tag{use a calculator}\\
    &=2\lrp{\frac{5}{6}-\ln 2}+\frac{1}{2}\ln 2\tag{use a calculator again...}\\
    &=\frac{5}{3}-2\ln 2 + \frac{1}{2}\ln 2\\
    &=\boxed{\frac{5}{3}-\frac{3}{2}\ln 2}
\end{align*}

\phantomsection
\addcontentsline{toc}{subsection}{2(b)}\textbf{(b)} $\displaystyle\F(x,y,z)=\lra{-4xy,8y,2}$, $\r(t)=\lra{t,t^2,1}$, $0\leq t\leq 2$

\Solution

Recall that the flow of the velocity field $\F$ along a curve is
\begin{align*}
    \text{flow}&=\int_C M\,dx + N\,dy+ P\,dz
\end{align*}
Note that this is \textit{one} of the many ways to define work. I am using the definition of work that is convenient for this problem. Plus, remembering the formula this way helps you remember the other identical formulas hahaha

If $\r(t)=\lra{t,t^2,1}$,
\begin{align*}
    \F\lrp{\r(t)}&=\lra{-4(t)(t^2),8t^2,2}=\lra{-4t^3,8t^2,2}\\
    \r'(t)&=\lra{1,2t,0}
\end{align*}
Let's find the flow for this problem.
\begin{align*}
    \text{flow}&=\int_0^2 \lrp{-4t^3}\lrp{1}+\lrp{8t^2}\lrp{2t}+(2)(0)\,dt\\
    &=\int_0^2 -4t^3 + 16t^3 + 0\,dt\\
    &=\int_0^2 12t^3\,dt\\
    &=\lrb{3t^4}_0^2\\
    &=3(2)^4-3(0)^4\\
    &=\boxed{48}
\end{align*}
\phantomsection
\addcontentsline{toc}{subsection}{2(c)}\textbf{(c)} $\displaystyle\F(x,y,z)=\lra{-y, x, 2}$, $\r(t)=\lra{-2\cos t, 2\sin t, 2t}$, $0\leq t \leq 2\pi$

\Solution

Recall that the flow of the velocity field $\F$ along a curve is
\begin{align*}
    \text{flow}&=\int_C M\,dx + N\,dy+ P\,dz
\end{align*}
Note that this is \textit{one} of the many ways to define work. I am using the definition of work that is convenient for this problem. Plus, remembering the formula this way helps you remember the other identical formulas hahaha

If $\r(t)=\lra{-2\cos t, 2\sin t, 2t}$,
\begin{align*}
    \F\lrp{\r(t)}&=\lra{-2\sin t, -2\cos t, 2}\\
    \r'(t)&=\lra{2\sin t, 2\cos t, 2}
\end{align*}
Let's find the flow for this problem.
\begin{align*}
    \text{flow}&=\int_0^{2\pi}\lrp{-2\sin t}\lrp{2\sin t}+\lrp{-2\cos t}\lrp{2\cos t}+\lrp{2}\lrp{2}\,dt\\
    &=\int_0^{2\pi} -4\sin^2 t -4\cos ^2 t+4\,dt\\
    &=\int_0^{2\pi}-4\lrp{\sin^2 t + \cos ^2 t}+4\,dt\\
    &=\int_0^{2\pi} -4+4\,dt\tag{$\sin^2 t +\cos^2 t$}\\
    &=\int_0^{2\pi} 0 \,dt\\
    &=\boxed{0}
\end{align*}
\phantomsection
\addcontentsline{toc}{section}{Problem 3}\textbf{Problem 3}

Find the flux of the fields $\F_1(x,y)=\lra{2x,-3y}$ and $\F_2(x,y)=\lra{2x,x-y}$ across the circle 

$\r(t)=\lra{a\cos t, a\sin t}$, $0\leq t \leq 2\pi$.

\Solution

Recall that the flux of a field $\F$ is
\begin{align*}
    \text{flux}&=\int_C M\,dy - N\,dx
\end{align*}

\phantomsection
\addcontentsline{toc}{subsection}{F1 Flux}

For $\F_1(x,y)=\lra{2x,-3y}$,

If $\r(t)=\lra{a\cos t,a\sin t}$,
\begin{align*}
    \F_1 \lrp{\r(t)}&=\lrp{2a\cos t,-3a\sin t}\\
    \r'(t)&=\lra{-a\sin t, a \cos t}
\end{align*}
Let's find the flux for $\F_1$.
\begin{align*}
    \text{flux}&=\int_0^{2\pi} \lrp{2a\cos t}\lrp{a \cos t}-\lrp{-3a\sin t}\lrp{-a \sin t}\,dt\\
    &=\int_0^{2\pi} 2a^2 \cos ^2 t - 3a^2 \sin ^2 t\,dt\\
    &=\int_0^{2\pi} 2a^2 \lrp{\frac{1}{2}(1+\cos 2t)}-3a^2\lrp{\frac{1}{2}(1-\cos 2t)}\,dt\tag{see note below}\\
    &=\int_0^{2\pi}a^2(1+\cos 2t) -\frac{3a^2}{2}(1-\cos 2t)\,dt\\
    &=\int_0^{2\pi} a^2 +a^2\cos 2t - \frac{3a^2}{2}+\frac{3a^2}{2}\cos 2t\,dt\\
    &=\int_0^{2\pi} -\frac{a^2}{2}+\frac{5a^2}{2}\cos 2t\,dt\\
    &=\lrb{-\frac{a^2}{2}t+\frac{5a^2}{4}\sin 2t}_0^{2\pi}\\
    &=\lrp{-\frac{a^2}{2}(2\pi) +\frac{5a^2}{4}\sin 4\pi}-\lrp{0+\frac{5a^2}{4}\sin 0}\\
    &=-a^2\pi
\end{align*}

\phantomsection
\addcontentsline{toc}{subsection}{F2 Flux}

For $\F_2(x,y)=\lra{2x,x-y}$,

If $\r(t)=\lra{a\cos t,a\sin t}$,
\begin{align*}
    \F_2 \lrp{\r(t)}&=\lrp{2a\cos t, a\cos t - a\sin t}\\
    \r'(t)&=\lrp{-a\sin t, a \cos t}
\end{align*}
Let's find the flux for $\F_2$.
\begin{align*}
    \text{flux}&=\int_0^{2\pi}\lrp{2a\cos t}\lrp{a\cos t}-\lrp{a\cos t-a\sin t}\lrp{-a \sin t}\,dt\\
    &=\int_0^{2\pi} 2a^2 \cos ^2 t - \lrp{-a^2\cos t \sin t + a^2 \sin ^2 t}\,dt\\
    &=\int_0^{2\pi} 2a^2\lrp{\frac{1}{2}\lrp{1+\cos 2t}}+a^2\lrp{\frac{1}{2}\sin 2t}-a^2\lrp{\frac{1}{2}\lrp{1-\cos 2t}}\,dt\tag{see note below}\\
    &=\int_0^{2\pi}a^2\lrp{1+\cos 2t}+\frac{a^2}{2}\sin 2t-\frac{a^2}{2}\lrp{1-\cos 2t}\,dt\\
    &=\int_0^{2\pi}a^2 +a^2\cos 2t + \frac{a^2}{2}\sin 2t - \frac{a^2}{2}+\frac{a^2}{2}\cos 2t\,dt\\
    &=\int_0^{2\pi} \frac{a^2}{2}+\frac{3a^2}{2}\cos 2t+\frac{a^2}{2}\sin 2t\,dt\\
    &=\lrb{\frac{a^2}{2}t+\frac{3a^2}{4}\sin 2t - \frac{a^2}{4}\cos 2t}_0^{2\pi}\\
    &=\lrp{\frac{a^2}{2}(2\pi)+\frac{3a^2}{4}\sin 2(2\pi)-\frac{a^2}{4}\cos 2(2\pi)}-\lrp{0 + \frac{3a^2}{4}\sin 0 - \frac{a^2}{4}\cos 0}\\
    &=\lrp{a^2\pi +0-\frac{a^2}{4}}-\lrp{0+0-\frac{a^2}{4}}\\
    &=\boxed{a^2\pi}
\end{align*}
\textbf{Note}

I use a number of trig tricks in this problem. They are
\begin{align*}
    \cos^2 t&=\frac{1}{2}\lrp{1+\cos 2t}\\
    2\cos t\sin t &=\sin 2t\\
    \sin^2 t &=\frac{1}{2}\lrp{1-\cos 2t}
\end{align*}

Our final answer is
\begin{subequations}
    \begin{empheq}[box=\widefbox]{align}
       \text{flux of }{\F_1}&: -a^2\pi \nonumber\\
       \text{flux of }{\F_2}&: a^2\pi \nonumber
    \end{empheq}
\end{subequations}
\phantomsection
\addcontentsline{toc}{section}{Problem 4}\textbf{Problem 4}

Find the circulation and flux of the fields $\F_1(x,y)=\lra{x^2,y^2}$ and $\F_2(x,y)=\lra{-y,x}$
around and across the closed path consisting of the semicircle $\r_1(t)=\lra{a\cos t, a\sin t}$, $0\leq t \leq \pi$ followed by the line segment $\r_2 (t)=\lra{t,0}$, $-a\leq t \leq a$.

(This is two different vector fields over the same closed curve.)

\Solution

Recall that the circulation and flux of a field $\F$ are
\begin{align*}
    \text{circulation}&=\int_C M\,dx + N\,dy\\
    \text{flux}&=\int_C M\,dy - N\,dx
\end{align*}
\phantomsection
\addcontentsline{toc}{subsection}{F1}

For $\F_1(x,y)=\lra{x^2,y^2}$,

If $\r_1(t)=\lra{a\cos t, a\sin t}$ and $\r_2 (t)=\lra{t,0}$, then
\begin{align*}
    \F_1\lrp{\r_1(t)}&=\lra{a^2\cos^2 t, a^2\sin ^2 t}\\
    \r_1'(t)&=\lra{-a \sin t ,a \cos t}\\
    \F_1\lrp{\r_2(t)}&=\lra{t^2,0}\\
    \r_2'(t)&=\lra{1,0}
\end{align*}
Let's find the circulation and flux for $\F_1$.
\phantomsection
\addcontentsline{toc}{subsubsection}{Circulation}
For circulation,
\begin{align*}
    \text{circulation}&=\underbrace{\int_0^\pi \lrp{a^2\cos ^2 t}\lrp{-a\sin t}+\lrp{a^2\sin ^2 t}\lrp{a \cos t}\,dt}_{\r_1(t)} + \underbrace{\int_{-a}^a \lrp{t^2}\lrp{1}+\lrp{0}\lrp{0}\,dt}_{\r_2(t)}\\
    &=\int_0^\pi -a^3 \sin t \cos ^2 t + a^3 \cos t \sin^2 t\,dt + \int_{-a}^a t^2\,dt\\
    &=\lrb{\frac{1}{3}a^3\cos^3 t + \frac{1}{3}a^3 \sin^3 t}_0^\pi + \lrb{\frac{1}{3}t^3}_{-a}^a\tag{or break up left integral and do u-sub(s)}\\
    &=\Bigg(\lrp{\frac{1}{3}a^3 \cos ^3 \pi + \frac{1}{3}a^3 \sin ^3 \pi}-\lrp{\frac{1}{3}a^3 \cos ^3 0 + \frac{1}{3}a^3 \sin ^3 0}\Bigg)+\Bigg(\frac{1}{3}a^3 - \lrp{\frac{1}{3}(-a)^3}\Bigg)\\
    &=\Bigg(\lrp{-\frac{1}{3}a^3}-\lrp{\frac{1}{3}a^3}\Bigg)+\Bigg(\frac{1}{3}a^3-\lrp{-\frac{1}{3}a^3}\Bigg)\\
    &=\lrp{-\frac{2}{3}a^3}+\lrp{\frac{2}{3}a^3}\\
    &=0
\end{align*}
\phantomsection
\addcontentsline{toc}{subsubsection}{Flux}
For flux
\begin{align*}
    \text{flux}&=\underbrace{\int_0^\pi \lrp{a^2\cos ^2 t}\lrp{a\cos t}-\lrp{a^2\sin ^2 t}\lrp{-a \sin t}\,dt}_{\r_1(t)} + \underbrace{\int_{-a}^a \lrp{t^2}\lrp{0}-\lrp{0}\lrp{1}\,dt}_{\r_2(t)}\\
    &= \int_0^\pi a^3\cos^3 t + a^3 \sin ^3 t\,dt +\int_{-a}^a 0\,dt\\
    &=a^3\int_0^\pi \cos^3 t \,dt + a^3\int_0^\pi  \sin^3 \,dt + 0\\
    &=a^3\underbrace{\int_0^\pi \lrp{1-\sin^2 t}\cos t \,dt}_{u} +a^3\underbrace{\int_0^\pi \lrp{1-\cos^2 t }\sin t\,dt}_{v}\\
    &=u=\sin t\hspace{2em}du=\cos t\,dt\\
    &u(0)=0\hspace{2em}u(\pi)=0\\
    &v=\cos t \hspace{2em}dv=-\sin t\,dt\\
    &v(0)=1\hspace{2em}v(\pi)=-1\\
    &=a^3\int_0^0 (1-u^2)\,du -a^3\int_1^{-1}(1-u^2)\,du\\
    &=0 + a^3\int_{-1}^1 1-u^2\,du\\
    &=a^3\lrb{u-\frac{1}{3}u^3}_{-1}^1\\
    &=a^3\Bigg(\lrp{1-\frac{1}{3}}-\lrp{-1-\frac{1}{3}(-1)}\Bigg)\\
    &=a^3\lrp{\frac{2}{3}+\frac{2}{3}}\\
    &=\frac{4}{3}a^3
\end{align*}

\phantomsection
\addcontentsline{toc}{subsection}{F2}
For $\F_2(x,y)=\lra{-y,x}$,

If $\r_1(t)=\lra{a\cos t, a\sin t}$ and $\r_2 (t)=\lra{t,0}$, then
\begin{align*}
    \F_2\lrp{\r_1(t)}&=\lra{-a\sin t, a \cos t}\\
    \r_1'(t)&=\lra{-a \sin t ,a \cos t}\\
    \F_2\lrp{\r_2(t)}&=\lra{0, t}\\
    \r_2'(t)&=\lra{1,0}
\end{align*}
Let's find the circulation and flux for $\F_2$.

\phantomsection
\addcontentsline{toc}{subsubsection}{Circulation}
For circulation,
\begin{align*}
    \text{circulation}&=\underbrace{\int_0^\pi \lrp{-a \sin t}\lrp{-a\sin t}+\lrp{a \cos t}\lrp{a \cos t}\,dt}_{\r_1(t)} + \underbrace{\int_{-a}^a \lrp{0}\lrp{1}+\lrp{t}\lrp{0}\,dt}_{\r_2(t)}\\
    &=\int_0^\pi a^2 \sin^2 t + a^2 \cos ^2 t \,dt + \int_{-a}^a 0\,dt\\
    &=\int_0^\pi a^2 \lrp{\sin^2 t +\cos^2 t}\, dt + 0\\
    &=\int_0^\pi a^2\,dt\tag{$\sin^2 t + \cos ^2 t = 1$}\\
    &=\lrb{a^2 t}_0^\pi\\
    &=a^2 \pi
\end{align*}
\phantomsection
\addcontentsline{toc}{subsubsection}{Flux}
For flux,
\begin{align*}
    \text{flux}&=\underbrace{\int_0^\pi \lrp{-a \sin t}\lrp{a\cos t}-\lrp{a \cos t}\lrp{-a \sin t}\,dt}_{\r_1(t)} + \underbrace{\int_{-a}^a \lrp{0}\lrp{0}-\lrp{t}\lrp{1}\,dt}_{\r_2(t)}\\
    &=\int_0^\pi -a^2 \sin t \cos t + a^2 \sin t \cos t\,dt + \int_{-a}^a -t\,dt\\
    &=\int_0^\pi 0 \,dt + \lrb{-\frac{1}{2}t^2}_{-a}^{a}\\
    &=0 + \lrp{\frac{1}{2}a^2 - \lrp{\frac{1}{2}(-a)^2}}\\
    &=\frac{1}{2}a^2 -\frac{1}{2}a^2\\
    &=0
\end{align*}
Our final answer is
\begin{subequations}
    \begin{empheq}[box=\widefbox]{align}
       \text{circulation of }{\F_1}&: 0\nonumber\\
       \text{flux of }{\F_1}&: \frac{4}{3}a^3\nonumber\\
       \text{circulation of }{\F_2}&: a^2\pi\nonumber\\
       \text{flux of }{\F_2}&: 0\nonumber
    \end{empheq}
\end{subequations}
\phantomsection
\addcontentsline{toc}{section}{Problem 5}\textbf{Problem 5}

Let $C$ be the ellipse in which the plane $2x+3y-z=0$ intersects the cylinder
$x^2+y^2=12$. Without evaluating the integral directly, show that the circulation of
$\F(x,y,z)=\lra{x,y,z}$ around $C$ is $0$.

\Solution

this problem makes me want to ff15. this is a badly written solution.

Recall that the circulation of the field $\F$ around $C$ is
\begin{align*}
    \text{circulation}&=\int_a^b \F(\r(t))\cdot\frac{d\r}{dt}\,dt
\end{align*}

Let $\r(t)=\lra{\sqrt{12}\cos t, \sqrt{12}\sin t, 2\sqrt{12}\cos t + 3\sqrt{12}\sin t}$, $0\leq t\leq 2\pi$ where $x$ and $y$ come from the cylinder $x^2+y^2=12$ and $z=2x+3y$.

If $\r(t)=\lra{\sqrt{12}\cos t, \sqrt{12}\sin t, 2\sqrt{12}\cos t + 3\sqrt{12}\sin t}$,
\begin{align*}
    \F\lrp{\r(t)}&=\lra{\sqrt{12}\cos t, \sqrt{12}\sin t, 2\sqrt{12}\cos t + 3\sqrt{12}\sin t}\\
    \r'(t)&=\lra{-\sqrt{12}\sin t, \sqrt{12}\cos t, -2\sqrt{12}\sin t + 3\sqrt{12}\cos t}
\end{align*}
Let's find $\F \cdot\dfrac{d\r}{dt}$.
\begin{align*}
    \F \cdot\frac{d\r}{dt}&=\lrp{\sqrt{12}\cos t}\lrp{-\sqrt{12}\sin t}+\lrp{\sqrt{12}\sin t}\lrp{\sqrt{12}\cos t}+\lrp{2\sqrt{12}\cos t + 3\sqrt{12}\sin t}\lrp{-2\sqrt{12}\sin t+3\sqrt{12}\cos t}\\
    &=0-48\sin t \cos t + 72 \cos^2t - 72 \sin^2 t + 108\sin t\cos t\\
    &=60\sin t\cos t + 72 \lrp{\cos ^2 t-\sin ^2 t}\\
    &=30\sin 2t + 72\cos 2t
\end{align*}
Let's find the circulation. Look at how we're not (\textit{technically}) directly evaluating the integral.
\begin{align*}
    \text{circulation}&=\int_0^{2\pi} 30\sin 2t + 72\cos 2t\,dt\\
    &=\int_0^{2\pi}30\sin 2t\,dt + \int_0^{2\pi}72 \cos 2t\,dt\\
    &=0 + 0\tag{periodicity of trig functions}\\
    &=0
\end{align*}
\qed
\end{document}
