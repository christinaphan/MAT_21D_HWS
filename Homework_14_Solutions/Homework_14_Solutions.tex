\documentclass{article}
\usepackage[utf8]{inputenc}
% are all of these packages really necessary?
% no.
% i'm just too lazy to only grab the packages i want for a specific
% document, so i just glob all of my most commonly used packages together
% this is bad practice.
\usepackage{amsmath,amsthm,amssymb,amsfonts, fancyhdr, color, comment, graphicx, environ, mdframed, soul, calc, enumitem, mdframed, xcolor, geometry, empheq, mathtools, tikz, pgfplots, caption, subcaption, hyperref}

\usetikzlibrary{external}
\tikzexternalize[prefix=tikz/,optimize command away=\includepdf]

%tikzpicture
\usepackage{tikz}
\usepackage{scalerel}
\usepackage{pict2e}
\usepackage{tkz-euclide}
\usetikzlibrary{calc}
\usetikzlibrary{patterns,arrows.meta}
\usetikzlibrary{shadows}
\usetikzlibrary{external}

%pgfplots
\usepackage{pgfplots}
\pgfplotsset{compat=newest}
\usepgfplotslibrary{statistics}
\usepgfplotslibrary{fillbetween}
\usepgfplotslibrary{polar}

\tikzset{external/export=true}
\pgfplotsset{
    standard/.style={
    axis line style = thick,
    trig format=rad,
    enlargelimits,
    axis x line=middle,
    axis y line=middle,
    enlarge x limits=0.15,
    enlarge y limits=0.15,
    every axis x label/.style={at={(current axis.right of origin)},anchor=north west},
    every axis y label/.style={at={(current axis.above origin)},anchor=south east}
    }
}
\newcommand*\widefbox[1]{\fbox{\hspace{2em}#1\hspace{2em}}}
% Command "alignedbox{}{}" for a box within an align environment
% Source: http://www.latex-community.org/forum/viewtopic.php?f=46&t=8144
\newlength\dlf  % Define a new measure, dlf
\newcommand\alignedbox[2]{
% Argument #1 = before & if there were no box (lhs)
% Argument #2 = after & if there were no box (rhs)
&  % Alignment sign of the line
{
\settowidth\dlf{$\displaystyle #1$}  
    % The width of \dlf is the width of the lhs, with a displaystyle font
\addtolength\dlf{\fboxsep+\fboxrule}  
    % Add to it the distance to the box, and the width of the line of the box
\hspace{-\dlf}  
    % Move everything dlf units to the left, so that & #1 #2 is aligned under #1 & #2
\boxed{#1 #2}
    % Put a box around lhs and rhs
}
}

\hypersetup{
    colorlinks=true,
    linkcolor=blue,
    filecolor=magenta,      
    urlcolor=cyan,
    pdftitle={Homework 14 Solutions},
    pdfpagemode=UseOutlines,
    bookmarksopen=true,
    pdfauthor={Christina Phan}
}
\newcommand{\lrp}[1]{\left( #1 \right)}
\newcommand{\abs}[1]{\left\vert #1 \right\vert}
\newcommand{\lra}[1]{\left\langle #1 \right\rangle}
\newcommand{\lrb}[1]{\left[ #1 \right]}
\newcommand{\norm}[1]{\left\lVert #1 \right\rVert}
\newcommand{\iintR}[0]{\iint\limits_{R}}
\renewcommand{\u}[0]{\mathbf{u}}
\renewcommand{\i}[0]{\mathbf{i}}
\renewcommand{\j}[0]{\mathbf{j}}
\renewcommand{\k}[0]{\mathbf{k}}
\newcommand{\T}[0]{\mathbf{T}}
\newcommand{\N}[0]{\mathbf{N}}
\newcommand{\B}[0]{\mathbf{B}}
\renewcommand{\r}[0]{\mathbf{r}}
\renewcommand{\a}[0]{\mathbf{a}}
\renewcommand{\v}[0]{\mathbf{v}}
\newcommand{\F}[0]{\mathbf{F}}

\geometry{letterpaper, portrait, margin=1in}
\renewcommand{\footrulewidth}{0.8pt}
\setlength\parindent{0pt}
\pagestyle{fancy}
\lhead{Christina Phan}
\rhead{MAT 21D} 
\chead{\textbf{Homework 14 Solutions}}

\newcommand{\Solution}{\textit{Solution}}
\pgfplotsset{compat=1.18}
\begin{document}
\textbf{Note}

For all of these problems we're just looking for $\textit{a}$ potential function, not a \textit{general} potential function. For this reason, we're going to assume $C=0$. No more $+C$ integral horror stories here...

\phantomsection
\addcontentsline{toc}{section}{Problem 1 (Parts)}\textbf{Problem 1 (Parts)}

Find a potential function $f$ for the vector field $\F$:

\phantomsection
\addcontentsline{toc}{subsection}{1(a)}\textbf{(a)} $\F\lrp{x,y,z}=\lra{2x,3y,4z}$

\Solution

If looks like $f=x^2+\frac{3}{2}y^2+2z^2$. We can verify this by checking if $\nabla f = \F$.
\begin{align*}
      \nabla f &=\lra{\frac{\partial f}{\partial x}, \frac{\partial f}{\partial y}, \frac{\partial f}{\partial z}}\\
      &=\lra{2x, 3y, 4z}\\
      &=\F
\end{align*}
 Since $\nabla f = \F$,
 \begin{equation*}
     \boxed{f=x^2+\frac{3}{2}y^2+2z^2}
 \end{equation*}
\phantomsection
\addcontentsline{toc}{subsection}{1(b)}\textbf{(b)} $\displaystyle\F(x,y,z)=\lra{e^{y+2x}, xe^{y+2x}, 2xe^{y+2z}}$

\Solution

If looks like $f = xe^{y+2z}$. We can verify this by checking if $\nabla f = \F$.
\begin{align*}
     \nabla f &=\lra{\frac{\partial f}{\partial x}, \frac{\partial f}{\partial y}, \frac{\partial f}{\partial z}}\\
     &=\lra{e^{y+2z}, xe^{y+2z}, 2xe^{y+2z}}\\
     &=\F
\end{align*}
Since $\nabla f = \F$,
\begin{equation*}
    \boxed{f = xe^{y+2z}}
\end{equation*}
\phantomsection
\addcontentsline{toc}{subsection}{1(c)}\textbf{(c)} $\displaystyle\F(x,y,z)=\lra{\ln x+\sec^2(x+y),\sec^2(x,y)+\frac{y}{y^2+z^2},\frac{z}{y^2+z^2}}$

\Solution

It looks like $f= x\ln x - x+\tan (x+y) + \frac{1}{2}\ln \left(y^2+z^2\right)$. We can verify this by checking if $\nabla f=\F$.
\begin{align*}
    \nabla f &=\lra{\frac{\partial f}{\partial x}, \frac{\partial f}{\partial y}, \frac{\partial f}{\partial z}}\\
    &=\lra{\ln x + \frac{x}{x}-1 + \sec^2 (x+y), \sec^2(x+y) +\frac{1}{2}\lrp{\frac{2y}{y^2+z^2}}, \frac{1}{2}\lrp{\frac{2z}{y^2+z^2}}}\\
    &=\lra{\ln x + 1 -1+\sec^2\lrp{x+y}, \sec^2(x+y)+\frac{y}{y^2+z^2},\frac{z}{y^2+z^2}}\\
    &=\lra{\ln x+\sec^2(x+y),\sec^2(x,y)+\frac{y}{y^2+z^2},\frac{z}{y^2+z^2}}\\
    &=\F
\end{align*}
Since $\nabla f = \F$,
\begin{align*}
    \boxed{f = x\ln x - x+\tan (x+y) + \frac{1}{2}\ln \left(y^2+z^2\right)}
\end{align*}
\newpage
\phantomsection
\addcontentsline{toc}{section}{Problem 2 (Parts)}\textbf{Problem 2 (Parts)}

The vector field in each of these integrals is conservative. Find a potential function for
the field and evaluate the integral:

\phantomsection
\addcontentsline{toc}{subsection}{2(a)}\textbf{(a)} $\displaystyle \int_{(0,2,1)}^{(1,\pi/2,2)} 2\cos y\,dx+\lrp{\frac{1}{y}-2x\sin y}\,dy+\frac{1}{z}\,dz$

\Solution

It looks like $\displaystyle f = 2x\cos y + \ln \left|y\right| + \ln \left|z\right|$. We can verify this by checking if $\nabla f = \F$ where $\displaystyle \F = \lra{2\cos y, \frac{1}{y}-2x\sin y, \frac{1}{z}}$.
\begin{align*}
     \nabla f &=\lra{\frac{\partial f}{\partial x}, \frac{\partial f}{\partial y}, \frac{\partial f}{\partial z}}\\
     &=\lra{2\cos y, -2x\sin y + \frac{1}{y}, \frac{1}{z}}\\
     &=\lra{2\cos y,  \frac{1}{y}-2x\sin y, \frac{1}{z}}\\
     &=\F
\end{align*}
If  $\displaystyle f = 2x\cos y + \ln \left|y\right| + \ln \left|z\right|$, then
\begin{align*}
    \int_{(0,2,1)}^{(1,\pi/2,2)} 2\cos y\,dx+\lrp{\frac{1}{y}-2x\sin y}\,dy+\frac{1}{z}\,dz &= \lrb{f(x,y,z)}_{(0,2,1)}^{(1,\pi/2, 2)}\\
    &=f(1,\pi/2, 2) - f(0,2,1)\\
    &=\lrp{2\cos \frac{\pi}{2} + \ln \left|\frac{\pi}{2}\right| + \ln \left|2\right|}-\lrp{0\cos 2 + \ln \left|2\right| + \ln \left|1 \right|}\\
    &=\lrp{0 + \ln \frac{\pi}{2}+ \ln 2} - \lrp{0 + \ln 2 + 0}\\
    &=\lrp{\ln \frac{2\pi}{2}}-\lrp{\ln 2}\tag{property of $\ln$}\\
    &=\ln \pi - \ln 2\\
    &=\boxed{\ln \frac{\pi}{2}}\tag{property of $\ln$}
\end{align*}
\phantomsection
\addcontentsline{toc}{subsection}{2(b)}\textbf{(b)} $\displaystyle \int_{(1,1,1)}^{(2,2,2)}\frac{1}{y}\,dx+\lrp{\frac{1}{z}-\frac{x}{y^2}}\,dy-\frac{y}{z^2}\,dz$

\Solution

It looks like $\displaystyle f = \frac{x}{y} + \frac{y}{z}$. We can verify this by checking if $\nabla f=\F$ where $\displaystyle \F=\lra{\frac{1}{y},\frac{1}{z}-\frac{x}{y^2},-\frac{y}{z^2}}$.
\begin{align*}
    \nabla f &=\lra{\frac{\partial f}{\partial x}, \frac{\partial f}{\partial y}, \frac{\partial f}{\partial z}}\\
    &=\lra{\frac{1}{y},-\frac{x}{y^2}+\frac{1}{z}, -\frac{y}{z^2}}\\
    &=\lra{\frac{1}{y},\frac{1}{z}-\frac{x}{y^2}, -\frac{y}{z^2}}\\
    &=\F
\end{align*}
If $\displaystyle f = \frac{x}{y} + \frac{y}{z}$, then
\begin{align*}
    \int_{(1,1,1)}^{(2,2,2)}\frac{1}{y}\,dx+\lrp{\frac{1}{z}-\frac{x}{y^2}}\,dy-\frac{y}{z^2}\,dz&= \lrb{f(x,y,z)}_{(1,1,1)}^{(2,2,2)}\\
    &=f(2,2,2)-f(1,1,1)\\
    &=\lrp{\frac{2}{2}+\frac{2}{2}}-\lrp{\frac{1}{1}+\frac{1}{1}}\\
    &=\lrp{1+1}-\lrp{1+1}\\
    &=\boxed{0}
\end{align*}
\phantomsection
\addcontentsline{toc}{subsection}{2(c)}\textbf{(c)} $\displaystyle \int_{(-1,-1,-1)}^{(2,2,2)}\frac{2x\,dx+2y\,dy+2z\,dz}{x^2+y^2+z^2}$

\Solution

Let's try rewriting this integral.
\begin{align*}
    \int_{(-1,-1,-1)}^{(2,2,2)}\frac{2x\,dx+2y\,dy+2z\,dz}{x^2+y^2+z^2} &=\int_{(-1,-1,-1)}^{(2,2,2)} \frac{2x}{x^2+y^2+z^2}\,dx+\frac{2y}{x^2+y^2+z^2}\,dy+\frac{2z}{x^2+y^2+z^2}\,dz
\end{align*}
It looks like $\displaystyle f=\ln\lrp{x^2+y^2+z^2}$. We can verify this by checking if $\nabla f=\F$ 

where $\displaystyle \F=\lra{\frac{2x}{x^2+y^2+z^2},\frac{2y}{x^2+y^2+z^2},\frac{2z}{x^2+y^2+z^2}}$.
\begin{align*}
    \nabla f &=\lra{\frac{\partial f}{\partial x}, \frac{\partial f}{\partial y}, \frac{\partial f}{\partial z}}\\
    &=\lra{\frac{2x}{x^2+y^2+z^2},\frac{2y}{x^2+y^2+z^2},\frac{2z}{x^2+y^2+z^2}}\\
    &=\F
\end{align*}
If $\displaystyle f=\ln\lrp{x^2+y^2+z^2}$, then
\begin{align*}
    \int_{(-1,-1,-1)}^{(2,2,2)}\frac{2x\,dx+2y\,dy+2z\,dz}{x^2+y^2+z^2}&=\lrb{f(x,y,z)}_{(-1,-1,-1)}^{(2,2,2)}\\
    &=f(2,2,2)-f(-1,-1,-1)\\
    &=\ln \lrp{2^2 + 2^2 + 2^2}-\ln \lrp{(-1)^2 + (-1)^2 +(-1)^2}\\
    &=\ln 12 - \ln 3\\
    &=\ln \frac{12}{3}\tag{property of $\ln$}\\
    &=\boxed{\ln 4}
\end{align*}
\phantomsection
\addcontentsline{toc}{section}{Problem 3}\textbf{Problem 3}

Let $\F=\nabla(x^3y^2)$ and let $C$ be the path in the $xy$-plane from $(-1,1)$ to $(1,1)$ consisting of the line segments from $(-1,1)$ and $(0,0)$ and from $(0,0)$ to $(1,1)$. Evaluate $\int_C \F\cdot d\r$ directly and check your answer using the Fundamental Theorem.

\Solution

\phantomsection
\addcontentsline{toc}{subsection}{Direct Method}\textbf{Direct Method}

If $\F=\nabla(x^3y^2)$, then
\begin{equation*}
    \F =\lra{\frac{\partial (x^3y^2)}{\partial x}, \frac{\partial (x^3y^2)}{\partial y}}=\lra{3x^2y^2, 2x^3y}
\end{equation*}
Let's break $C$ up into two curves, $C_1$ and $C_2$. We'll have $C_1$ be the line segment from $(-1,1)$ to $(0,0)$ and $C_2$ be the line segment from $(0,0)$ to $(1,1)$.

Let's find the parameterization, $r(t)$, for both of these curves.
\begin{align*}
    C_1: \r_1(t)&=\lra{-1,1}+\lra{0 - (-1), 0 - 1}t = \lra{-1, 1}+\lra{ t, -t}=\lra{-1 + t, 1 -t}\tag{$0\leq t \leq 1$}\\
    C_2: \r_2(t)&=\lra{0,0}+ \lra{1 - 0, 1 - 0}t=\lra{0,0}+\lra{t, t}=\lra{t, t}\tag{$0\leq t \leq 1$}
\end{align*}
We can verify that the bounds of $t$ on both of these curves are $0\leq t\leq 1$ since at $t=0$, $\r_1(0)=\lra{-1,1}$ and $\r_2(0)=\lra{0,0}$ and at $t=1$, $\r_1(1)=\lra{0,0}$ and $\r_2(1)=\lra{1,1}$.

If $\r_1(t)=\lra{-1 + t, 1 - t}$ and $\r_2(t)=\lra{t,t}$,
\begin{align*}
    \F\lrp{\r_1(t)}&=\lra{3(-1+t)^2(1-t)^2, 2(-1+t)^3(1-t)}\\
    &=\lra{3\Big((-1)(1-t)\Big)^2(1-t)^2, 2\Big((-1)(1-t)\Big)^3(1-t)}\\
    &=\lra{3(-1)^2(1-t)^2(1-t)^2, 2(-1)^3(1-t)^3(1-t)}\\
    &=\lra{3(1-t)^4,-2(1-t)^4}\\
    \r_1'(t)&=\lra{1,-1}\\
    \F\lrp{\r_2(t)}&=\lra{3(t)^2(t)^2, 2(t)^3(t)}=\lra{3t^4, 2t^4}\\
    \r_2'(t)&=\lra{1,1}
\end{align*}
Let's evaluate the integral.
\begin{align*}
    \int_C \F\cdot d\r &=\int_{C_1} \F\cdot d\r_1 + \int_{C_2}\F\cdot d\r_2\\
    &=\int_0^1 \lra{3(1-t)^4, -2(1-t)^4}\cdot \lra{1,-1}\,dt+\int_0^1 \lra{3t^4,2t^4}\cdot \lra{1,1}\,dt\\
    &=\int_0^1 3(t-1)^4 + 2(1-t)^4\,dt+\int_0^1 3t^4 + 2t^4\,dt\\
    &=\int_0^1 5(t-1)^4\,dt +\int_0^1 5t^4\,dt\\
    &=\lrb{(t-1)^5}_0^1 + \lrb{t^5}_0^1\tag{or do u-sub $u=t-1$}\\
    &= \Big((1-1)^5 - (0-1)^5\Big)+\lrp{1^5 - 0^5}\\
    &=\Big(0-(-1)\Big)+\lrp{1-0}\\
    &=\lrp{1}+\lrp{1}\\
    &=\boxed{2}
\end{align*}
\phantomsection
\addcontentsline{toc}{subsection}{Fundamental Theorem Method}\textbf{Fundamental Theorem Method}

Assuming $\F = \nabla f $ is a conservative vector field, then
\begin{align*}
    \int_{(-1,1)}^{(1,1)} \F \cdot d\r&= \lrb{f(x,y)}_{(-1,1)}^{(1,1)}\\
    &=f(1,1)-f(-1,1)\\
    &=\Big((1)^3(1)^2\Big)-\Big((-1)^3(1)^2\Big)\tag{remember: $f(x,y)=x^3y^2$}\\
    &=\lrp{1}-\lrp{-1}\\
    &=\boxed{2}
\end{align*}
\newpage
\phantomsection
\addcontentsline{toc}{section}{Problem 4}\textbf{Problem 4}

Suppose that $\F=\nabla f$ is a conservative vector field and $\displaystyle g(x,y,z)=\int_{(0,0,0)}^{(x,y,z)}\F\cdot d\r$. Show that $\nabla g=\F$.

\Solution

If $\F=\nabla f$ is a conservative vector field and $\displaystyle g(x,y,z)=\int_{(0,0,0)}^{(x,y,z)}\F\cdot d\r$, then
\begin{align*}
    g(x,y,z)&=\int_{(0,0,0)}^{(x,y,z)}\F\cdot d\r\\
    &=\int_{(0,0,0)}^{(x,y,z)} \nabla f \cdot d\r\\
    &=f(x,y,z)-f(0,0,0)
\end{align*}
If $g(x,y,z)=f(x,y,z)-f(0,0,0)$, then
\begin{align*}
    \nabla g &=\lra{\frac{\partial f}{\partial x}- 0, \frac{\partial f}{\partial y}- 0, \frac{\partial f}{\partial z}- 0}\tag{$f(0,0,0)$ is just some constant}\\
    &=\lra{\frac{\partial f}{\partial x}, \frac{\partial f}{\partial y}, \frac{\partial f}{\partial z}}\\
    &=\nabla f\\
    &=\F
\end{align*}
\qed
\end{document}
